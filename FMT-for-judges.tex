\documentclass[12pt]{article}
\usepackage[utf8]{inputenc}
\usepackage[english,russian]{babel}
\usepackage{amsmath}
\usepackage{amssymb}
\usepackage{geometry,indentfirst,color}
\usepackage[pdftex]{graphicx}
\usepackage{sidecap}
\geometry{top=0.5cm} %поле сверху
\geometry{bottom=0.5cm} %поле снизу
\geometry{left=1.5cm} %поле справа
\geometry{right=1.5cm}
\usepackage{wrapfig}
\usepackage{epigraph}
%\pagestyle{empty}
\DeclareGraphicsExtensions{.pdf,.png,.jpg}
\DeclareMathOperator{\Tr}{Tr}
\setlength{\textheight}{24cm}\newcounter{nnn}\setlength{\topmargin}{-20mm}
\begin{document}
\section*{Памятка для судей}
\subsection*{Основной этап}
Длится 20 минут, {\bf зондера полноценно участвуют в судействе}. Один из судей отвечает за ведение протокола и следит за временем заявок и.\,т.\,д.
Порядок действий при заявке задачи:
\begin{enumerate}
	\item Спросить вольных стрелков, располагают ли они решением, и взять его при наличии.
	\item Спросить ответ и зафиксировать его.
	\item Вызвать докладчика и оппонента, выслушать доклад.
	\item Спросить оппонента: <<Есть ли у оппонента {\bf вопросы} по предъявленному решению?>>. Следить, чтобы он не тянул время, есть возможность прервать фазу вопросов, если оппонент тянет время.
	\item <<Есть ли у оппонента {\bf замечания} к предъявленному решению?>> Оппонент делает замечания.
	\item Все садятся по местам, принять решение о баллах за решение и оппонирование и снятии задачи или переходе подачи.
\end{enumerate}

\subsection*{Обмен ударами}
Длится 10 минут (решение задач), зондера полноценно участвуют в судействе.
Порядок действий при заявке задачи:
\begin{enumerate}
	\item Судьи могут задавать вопросы как докладчику своей задачи, так и докладчику от команды соперника.
	\item Время на рассказ своей задачи --- 3 минуты, {\bf включая вопросы судей по решению}.
	\item Если при решении используется материал/теоремы, выходящий за рамки школьной программы, судьи просят доказать эти теоремы в процессе изложения решения своей задачи. 
	\item За незнание свой задачи можно штрафовать на 1 и 2 балла. Помните, что штраф на 1 балл также возможен, если <<незнание>> небольшое или рассказчик совсем немного не уложился во время.
	\item Решение задачи соперника также награждается 1 или 2 баллами.
\end{enumerate}
\end{document} 
